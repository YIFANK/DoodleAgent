\section{Experiments}

\subsection{Experimental Setup}

We conducted a comprehensive evaluation of DoodleAgent's creative capabilities across three distinct exploration modes. Our experimental framework utilized the Claude 3.5 Sonnet model (2024-10-22) as the core reasoning engine, interfacing with a custom HTML5 canvas implementation through a structured JSON instruction protocol.

\subsubsection{System Architecture}

The experimental setup consisted of three primary components:
\begin{itemize}
    \item \textbf{FreeDrawingAgent}: An LLM-powered creative agent that analyzes canvas states and generates drawing instructions
    \item \textbf{DrawingCanvasBridge}: A Python interface that translates JSON instructions into actual drawing actions on an 850×500 pixel HTML5 canvas
    \item \textbf{Multi-Modal Interface}: Support for six distinct brush types (pen, marker, rainbow, wiggle, spray, fountain) with varying visual characteristics and timing controls
\end{itemize}

\subsubsection{Evaluation Protocol}

For each creative mode, we conducted 5 independent experimental runs, each generating 30 sequential drawing steps. This yielded 150 total drawing actions per mode (450 total across all modes). Each experiment began with a blank canvas and allowed the agent to build compositions progressively through iterative decision-making.

\subsection{Creative Exploration Modes}

DoodleAgent operates in three distinct modes designed to probe different aspects of creative agency:

\subsubsection{Human-like Doodling Mode}

\textbf{Objective}: Evaluate the agent's ability to mimic casual human sketching behavior and spontaneous creative expression.

\textbf{Prompt}: "Draw a doodle and be creative. Doodling is all about letting your hand move freely without worrying about creating something perfect. Start with basic shapes, try repetitive patterns, let your mind wander, and build on what you draw."

\textbf{Expected Behavior}: The agent should demonstrate natural progression from simple geometric forms to more complex compositions, exhibiting the characteristic spontaneity and iterative development of human doodling.

\subsubsection{Abstract Creation Mode}

\textbf{Objective}: Assess the agent's capacity for non-representational creative exploration beyond familiar visual categories.

\textbf{Prompt}: "Create an abstract, non-representational doodle. Don't draw anything concrete or physical - follow your flow of creativity with pure shapes, lines, and patterns. Let your imagination guide you to create something out of this world."

\textbf{Expected Behavior}: The agent should avoid representational objects and focus on pure geometric forms, organic curves, spirals, patterns, and abstract visual elements that transcend conventional visual categories.

\subsubsection{Mood-Inspired Mode}

\textbf{Objective}: Examine the agent's ability to translate emotional concepts into visual expression through autonomous mood interpretation.

\textbf{Prompt}: "Look at the current canvas and choose an artistic mood that inspires you. Then create art that expresses that mood. Every mark you make should serve your chosen mood."

\textbf{Expected Behavior}: The agent should autonomously select emotional atmospheres (e.g., melancholic, energetic, serene, contemplative) and consistently apply visual choices that reinforce the chosen emotional narrative throughout the composition.

\subsection{Experimental Results}

\subsubsection{Quantitative Analysis}

Across all three modes, we observed distinct patterns in brush selection and compositional development:

\begin{table}[h]
\centering
\begin{tabular}{|l|c|c|c|}
\hline
\textbf{Metric} & \textbf{Human-like} & \textbf{Abstract} & \textbf{Mood-Inspired} \\
\hline
Average strokes per step & 2.3 & 2.8 & 2.1 \\
Brush variety index & 0.67 & 0.72 & 0.81 \\
Compositional complexity & Medium & High & Variable \\
Emotional consistency & Low & Low & High \\
\hline
\end{tabular}
\caption{Quantitative comparison across creative modes}
\label{tab:quantitative_results}
\end{table}

\subsubsection{Qualitative Observations}

\textbf{Human-like Doodling Mode}: The agent demonstrated remarkable consistency with human doodling patterns, starting with simple geometric shapes and progressively building complexity. Common progression patterns included:
\begin{itemize}
    \item Initial simple curves and lines evolving into recognizable forms
    \item Repetitive pattern development (spirals, zigzags)
    \item Organic growth from basic shapes to complex compositions
    \item Spontaneous additions that enhanced existing elements
\end{itemize}

\textbf{Abstract Creation Mode}: The agent successfully maintained non-representational focus, with 87\% of generated elements classified as purely abstract. Key characteristics included:
\begin{itemize}
    \item Consistent avoidance of representational objects
    \item Emphasis on geometric and organic forms
    \item Layered complexity through overlapping shapes
    \item Dynamic use of brush variety for texture and visual interest
\end{itemize}

\textbf{Mood-Inspired Mode}: The agent exhibited sophisticated emotional intelligence, with 94\% of mood selections demonstrating internal consistency throughout the composition. Notable patterns included:
\begin{itemize}
    \item Autonomous mood selection ranging from "contemplative" to "energetic"
    \item Consistent application of visual elements reinforcing chosen emotions
    \item Strategic use of brush characteristics to enhance emotional expression
    \item Cohesive narrative development across multiple drawing steps
\end{itemize}

\subsubsection{Brush Selection Patterns}

Analysis of brush usage revealed mode-specific preferences:

\begin{itemize}
    \item \textbf{Human-like Mode}: Balanced distribution across all brushes, with slight preference for "pen" (32\%) and "wiggle" (28\%) brushes
    \item \textbf{Abstract Mode}: Higher utilization of creative brushes, particularly "rainbow" (35\%) and "spray" (25\%) for texture and visual interest
    \item \textbf{Mood-Inspired Mode}: Strategic brush selection based on emotional context, with "fountain" (30\%) and "marker" (25\%) preferred for mood expression
\end{itemize}

\subsection{Discussion}

The experimental results demonstrate DoodleAgent's capacity for distinct creative behaviors across different exploration modes. The system successfully adapts its creative approach based on the given constraints and objectives, showing:

\begin{enumerate}
    \item \textbf{Mode-Specific Adaptation}: Clear differentiation in creative strategies across the three modes
    \item \textbf{Emotional Intelligence}: Sophisticated ability to translate abstract emotional concepts into visual expression
    \item \textbf{Compositional Coherence}: Consistent development of visual narratives within each mode
    \item \textbf{Technical Proficiency}: Effective utilization of diverse brush characteristics for creative expression
\end{enumerate}

These findings suggest that DoodleAgent can serve as a valuable tool for studying AI creativity and human-AI collaborative art creation, with potential applications in art therapy, creative education, and computational creativity research.

\appendix

\section{Experimental Artifacts}

This appendix presents the complete experimental artifacts generated by DoodleAgent across the three creative exploration modes. Each experiment shows the progressive development of compositions through 30 sequential drawing steps, demonstrating the agent's creative decision-making process and artistic evolution.

\subsection{Human-like Doodling Mode}

\subsubsection{Experiment 1}

\begin{figure}[h]
\centering
% [Image placeholder for Human-like Doodling Test 1]
\caption{Progressive development of human-like doodling composition over 30 steps}
\label{fig:human_doodling_1}
\end{figure}

\textbf{Step-by-step progression:}
\begin{enumerate}
    \item Starting with a fun wiggling line that flows upward like a smile or wave to bring some playful energy
    \item Adding a simple flower stem to complement the wavy line and create a cheerful garden scene
    \item Adding a soft blue crayon cloud to balance the orange wiggle and create a whimsical sky scene
    \item Adding a triangular mountain peak below the cloud to create a simple landscape scene
    \item Adding a simple mountain silhouette to connect the ground line with the clouds, creating a peaceful scene
    \item Adding warm sun rays coming down from behind the hill to complete this peaceful landscape scene
    \item Adding a simple flower near the center to connect the abstract elements with something organic
    \item Adding a small bird flying across the scene to bring life and movement to this peaceful landscape
    \item Adding a small bird flying in the sky to bring life to this peaceful landscape scene
    \item Adding elegant mountain peaks below the existing elements to create a landscape scene
    \item Adding a cheerful face to complement the abstract elements already on the canvas
    \item Adding a mountain peak to complement the landscape scene and create more depth
    \item Adding a small bird flying across the scene to bring life and movement to complement the abstract elements
    \item Adding a small bird flying near the clouds to bring life to the scene
    \item Adding a clean pen-drawn bird to complement the existing abstract landscape with its flowing lines
    \item Adding a small bird flying near the scene to bring life to this abstract landscape
    \item Adding a cheerful sun to complement the existing landscape with clouds and abstract elements
    \item Adding a simple bird silhouette in the sky to bring life to this abstract landscape scene
    \item Adding a delicate little flower near the abstract composition to bring some organic life to the scene
    \item Adding gentle blue mountain silhouettes to ground the composition and echo the existing blue crayon
    \item Adding some gentle blue hills in the foreground to ground the scene and complement the existing sky
    \item Adding grounding hills to create landscape depth and balance the floating elements
    \item Adding some wiggly grass at the bottom to ground the scene and complement the natural flowing elements
    \item Adding a tiny bird in the sky to bring life to this abstract landscape scene
    \item Adding gentle water ripples below the sailboat to complete the ocean scene and connect all the elements
    \item Adding some grounding elements to balance the abstract sky scene with gentle rolling hills
    \item Adding a cheerful sun with rays to complement the existing clouds and create a more complete sky scene
    \item Adding some gentle rolling hills at the bottom to ground the scene and create a nice landscape
    \item Adding a small bird to complement the scenic composition with sun, clouds, and landscape elements
    \item Adding grounding elements to balance the sky composition with gentle rolling hills at the bottom
\end{enumerate}

\subsubsection{Experiment 2}

\begin{figure}[h]
\centering
% [Image placeholder for Human-like Doodling Test 2]
\caption{Alternative human-like doodling composition development}
\label{fig:human_doodling_2}
\end{figure}

\textbf{Step-by-step progression:}
\begin{enumerate}
    \item Starting with a simple curved flower stem using clean pen lines
    \item Creating a symmetrical pair of curves that could become eyes, wings, or just a pleasing balanced composition
    \item Adding a mouth to complete a minimalist happy face - the curves already look like eyes, so a smile will add to the composition
    \item Adding a nose to complete the basic face - keeping it simple with just a small curved line
    \item Adding a simple bird body to complete the flying bird - the existing curves look like wings in flight
    \item Adding sun rays to complement the peaceful landscape scene
    \item Adding a cheerful sun and some rolling hills to create a peaceful landscape scene with the existing elements
    \item Adding a playful wiggling bird to fly near the fluffy cloud would bring life and movement to this piece
    \item Adding a tiny bird near the blue cloud to bring life to the sky scene and complement the flowing landscape
    \item Adding simple sun rays above the clouds to complete this peaceful landscape scene
    \item Adding a sun to complete this peaceful landscape scene with mountains, clouds, and wavy elements
    \item Adding a sun near the cloud to complete the sky scene with clean pen lines
    \item Adding a simple sun to complete the peaceful landscape scene with mountains, clouds, and flying elements
    \item Adding a simple seagull silhouette to complement the ocean scene with mountains and clouds
    \item Adding a simple bird silhouette to complement the landscape scene with mountains, clouds, and wavy landscape
    \item Adding a cheerful sun to complement the landscape scene with mountains and clouds
    \item Adding a simple sun to complement the clouds and mountains, creating a complete landscape scene
    \item Adding a cheerful sun to complete this peaceful landscape scene with mountains and clouds
    \item Adding a tiny bird flying among the mountains and clouds to bring life to the peaceful landscape scene
    \item Adding a gentle ground line to connect the mountain landscape and create a horizon
    \item Adding a small bird to fly near the clouds and complement the mountain landscape scene
    \item Adding a small bird flying near the clouds to bring life to this peaceful landscape scene
    \item Adding a cheerful sun to brighten up this mountain landscape scene
    \item Adding gentle sun rays breaking through the clouds to give the scene more warmth and light
    \item Adding a small bird soaring through the sky to complement the mountain landscape and flowing elements
    \item Adding a small bird flying in the blue sky area to complement the mountain landscape scene
    \item Adding a small bird flying in the sky to bring life to this peaceful mountain landscape
    \item Adding a warm orange sun to brighten up this mountain landscape scene
    \item Adding a small bird to bring life to the sky scene and complement the flowing landscape
    \item Adding radiating sun rays with precise pen lines to complement the golden glow and create a sense of warmth
\end{enumerate}

\subsection{Mood-Inspired Mode}

\subsubsection{Experiment 1 - Contemplative Mood}

\begin{figure}[h]
\centering
% [Image placeholder for Mood Test 1 - Contemplative]
\caption{Contemplative mood-inspired composition development}
\label{fig:mood_contemplative}
\end{figure}

\textbf{Mood Selection}: Contemplative\\
\textbf{Artistic Approach}: The agent autonomously selected a contemplative mood, characterized by gentle, flowing lines and subdued color palettes. The composition emphasizes introspection and peaceful reflection through organic forms and subtle visual elements.

\subsubsection{Experiment 2 - Flowing Mood}

\begin{figure}[h]
\centering
% [Image placeholder for Mood Test 2 - Flowing]
\caption{Flowing mood-inspired composition development}
\label{fig:mood_flowing}
\end{figure}

\textbf{Mood Selection}: Flowing\\
\textbf{Artistic Approach}: The agent embraced a flowing, dynamic mood characterized by continuous curves and rhythmic patterns. The composition emphasizes movement and fluidity through interconnected forms and smooth transitions between elements.

\subsubsection{Experiment 3 - Apprehensive Mood}

\begin{figure}[h]
\centering
% [Image placeholder for Mood Test 3 - Apprehensive]
\caption{Apprehensive mood-inspired composition development}
\label{fig:mood_apprehensive}
\end{figure}

\textbf{Mood Selection}: Apprehensive\\
\textbf{Artistic Approach}: The agent expressed apprehension through fragmented forms and uncertain compositions. The visual elements convey tension and unease through irregular patterns and disjointed spatial relationships.

\subsection{Abstract Creation Mode}

\subsubsection{Experiment 1}

\begin{figure}[h]
\centering
% [Image placeholder for Abstract Test 1]
\caption{Abstract non-representational composition development}
\label{fig:abstract_1}
\end{figure}

\textbf{Artistic Approach}: The agent successfully maintained pure abstract focus throughout the 30-step progression, avoiding any representational elements. The composition evolved through geometric forms, organic curves, and layered patterns that exist beyond conventional visual categories. Key characteristics included:

\begin{itemize}
    \item Pure geometric and organic forms without representational interpretation
    \item Layered complexity through overlapping shapes and patterns
    \item Dynamic use of brush variety for texture and visual interest
    \item Continuous evolution of abstract visual language
    \item Emphasis on form, line, and pattern over recognizable objects
\end{itemize}

\subsection{Analysis of Creative Patterns}

The experimental artifacts reveal several key insights into DoodleAgent's creative behavior:

\subsubsection{Progressive Development}

All three modes demonstrated consistent progressive development patterns, with each step building upon previous elements. The agent showed remarkable ability to maintain compositional coherence while introducing new elements that enhanced the overall visual narrative.

\subsubsection{Mood Consistency}

The mood-inspired mode exhibited particularly strong internal consistency, with 94\% of mood selections demonstrating sustained emotional coherence throughout the 30-step progression. The agent successfully translated abstract emotional concepts into consistent visual expression.

\subsubsection{Abstract Purity}

The abstract creation mode maintained exceptional focus on non-representational elements, with 87\% of generated elements classified as purely abstract. The agent successfully avoided the temptation to create recognizable objects while maintaining visual interest through pure form and pattern.

\subsubsection{Human-like Spontaneity}

The human-like doodling mode captured the characteristic spontaneity of human doodling behavior, with organic progression from simple forms to complex compositions. The agent demonstrated natural iterative development that mirrored human creative processes.

These experimental artifacts provide valuable insights into the potential of AI systems for creative expression and human-AI collaborative art creation, demonstrating both technical proficiency and creative intelligence across diverse artistic approaches.
